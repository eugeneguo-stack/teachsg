\documentclass[10pt,landscape]{article}
\usepackage{multicol}
\usepackage{calc}
\usepackage{ifthen}
\usepackage{gensymb}
\usepackage{tabularx,array,booktabs} % for stretching tables to page width
\newcolumntype{Y}{>{\centering\arraybackslash}X} % for centering tables in tabularx
\newcolumntype{N}{@{}m{0pt}@{}} % for vertical space in tabularx
\newcommand{\heading}[1]{\bfseries\begin{tabular}{@{}c@{}} #1 \end{tabular}} % for a header in tables
\def\tabularxcolumn#1{m{#1}} % for vertically centering text in tables
\usepackage{float}
\usepackage[landscape]{geometry}
\usepackage{hyperref}
\usepackage{tikz} % for rainbow method in expansion
\usetikzlibrary{calc,shapes} % for rainbow method in expansion
\usetikzlibrary{quotes,angles} % for angles
\usepackage{pstricks-add} % for geogebra
\usepackage{amsmath,amssymb}
\usepackage{esvect} % for vectors
\usepackage{undertilde} % for vectors
\usepackage{fancyhdr}
\usepackage{tcolorbox}% for framed rounded boxes
\tcbset{colframe=black,colback=white,colupper=black,
fonttitle=\bfseries,nobeforeafter,center title,size=small}
\tcbuselibrary{breakable}
\usepackage{framed}% for framed standard boxes

\linespread{1.2}


\newenvironment{Figure}
  {\par\medskip\noindent\minipage{\linewidth}}
  {\endminipage\par\medskip}




\newenvironment{cframed}
{
\begin{framed}\begin{center}
}
{
\end{center}\end{framed}
}




% This sets page margins to .5 inch if using letter paper, and to 1cm
% if using A4 paper. (This probably isn't strictly necessary.)
% If using another size paper, use default 1cm margins.
\ifthenelse{\lengthtest { \paperwidth = 11in}}
	{ \geometry{top=.5in,left=.5in,right=.5in,bottom=.7in} }
	{\ifthenelse{ \lengthtest{ \paperwidth = 297mm}}
		{\geometry{top=1cm,left=1cm,right=1cm,bottom=1cm} }
		{\geometry{top=1cm,left=1cm,right=1cm,bottom=1cm} }
	}

% Turn off header and footer
%\pagestyle{empty}


 

% Redefine section commands to use less space
\makeatletter
\renewcommand{\section}{\@startsection{section}{1}{0mm}%
                                {-1ex plus -.5ex minus -.2ex}%
                                {0.5ex plus .2ex}%x
                                {\normalfont\large\bfseries}}
\renewcommand{\subsection}{\@startsection{subsection}{2}{0mm}%
                                {-1explus -.5ex minus -.2ex}%
                                {0.5ex plus .2ex}%
                                {\normalfont\normalsize\bfseries}}
\renewcommand{\subsubsection}{\@startsection{subsubsection}{3}{0mm}%
                                {-1ex plus -.5ex minus -.2ex}%
                                {1ex plus .2ex}%
                                {\normalfont\small\bfseries}}
\makeatother

% Define BibTeX command
\def\BibTeX{{\rm B\kern-.05em{\sc i\kern-.025em b}\kern-.08em
    T\kern-.1667em\lower.7ex\hbox{E}\kern-.125emX}}

% Don't print section numbers
%\setcounter{secnumdepth}{0}
\setcounter{secnumdepth}{1}


\setlength{\parindent}{0pt}
\setlength{\parskip}{0pt plus 0.5ex}


\pagestyle{fancy}
\fancyhf{} % sets both header and footer to nothing
\renewcommand{\headrulewidth}{0pt}
\lfoot{\sc{\copyright\ 2019 Eugene Guo Youjun\\All rights reserved}}
\cfoot{\sc{Page \thepage}}
\rfoot{\sc{For Hai Sing Catholic School}}

% -----------------------------------------------------------------------

\begin{document}

\raggedright
\footnotesize
\begin{multicols}{3}


% multicol parameters
% These lengths are set only within the two main columns
\setlength{\columnseprule}{0.25pt}
\setlength{\premulticols}{1pt}
\setlength{\postmulticols}{1pt}
\setlength{\multicolsep}{1pt}
\setlength{\columnsep}{2pt}

\begin{center}
     \Large{\textbf{Elementary Mathematics Notes}} \\
    \small{Reproduced from \url{http://teach.sg}}
\end{center}

\section{Numbers \& Their Operations}

\subsection{Types of numbers}

Integers ($\mathbb{Z}$): $..., -3, -2, -1, 0, 1, 2, 3, 4, ...$\\
Prime: integers that are divisible by 1 and itself only, smallest prime number is 2\\
Rational numbers ($\mathbb{Q}$) $\dfrac{\text{integer}}{\text{integer}}$: $\dfrac{4}{7}, -3\dfrac{1}{8}, 0.3, 2.\dot{6}\dot{5}, 92, \sqrt{16}$\\
Irrational numbers: $\pi, \sqrt{2}, e$\\
Real numbers ($\mathbb{R}$): all numbers

\subsection{Standard form}

$A\times10^n$, where $n$ is an integer, and $1\leq A<10$

\subsection{SI prefix}





\begin{table}[H]
\centering
\begin{tabularx}{0.2\textwidth}{|Y|Y|}
\hline
\heading{Prefix} & \heading{$10^n$}\\ \hline
pico & $10^{-12}$ \\\hline
nano & $10^{-9}$ \\\hline
micro & $10^{-6}$ \\\hline
milli & $10^{-3}$ \\\hline
kilo & $10^3$ \\\hline
mega & $10^6$ \\\hline
giga & $10^9$ \\\hline
tera & $10^{12}$\\\hline
\end{tabularx}
\end{table}


\subsection{Indices}

$1.\,a^m\times a^n=a^{m+n}$\\
$2.\,a^m\div a^n=a^{m-n}$\\
$3.\,(a^m)^n=a^{mn}$\\
$4.\,(ab)^m=a^mb^m$\\
$5.\,\left(\dfrac{a}{b}\right)^n=\dfrac{a^n}{b^n}$\\
$6.\,a^{-n}=\dfrac{1}{a^n}$\\
$7.\,a^0=1$\\
$8.\,a^\frac{1}{n}=\sqrt[n]{a}$\\
$9.\,a^\frac{m}{n}=(\sqrt[n]{a})^m$


\section{Ratio \& Proportion}

\subsection{Map scale}
Length scale = $1:r$\\
Area scale = $1:r^2$


\section{Percentage}

$\text{Percentage increase / decrease}=\dfrac{\text{increase / decrease}}{\text{original}}\times 100\%$

\section{Rate \& Speed}

\begin{figure}[H]
\centering
\psset{xunit=0.7cm,yunit=0.7cm,algebraic=true,dimen=middle,dotstyle=o,dotsize=3pt 0,linewidth=0.8pt,arrowsize=3pt 2,arrowinset=0.25}
\psscalebox{0.5 0.5}{
\begin{pspicture*}(-6.08555548614,-1)(6,5.3)
\pspolygon[linecolor=black](0.,5.)(2.88675,0.)(-2.88675,0.)
\psline[linecolor=black](0.,5.)(2.88675,0.)
\psline[linecolor=black](2.88675,0.)(-2.88675,0.)
\psline[linecolor=black](-2.88675,0.)(0.,5.)
\psline(-1.49193285585,2.41589528734)(1.48976761918,2.41964558901)
\psline(0.,2.42)(0.,0.)
\rput[tl](-0.3,3.7){\Large{D}}
\rput[tl](-1.4,1.3){\Large{S}}
\rput[tl](0.8,1.3){\Large{T}}
\end{pspicture*}
}
\end{figure}

$\text{Average speed}=\dfrac{\text{total distance}}{\text{total time}}$


\section{Algebraic Expressions \& Formulae}

\subsection{$n^{th}$ term}
$a+(n-1)d$


\subsection{Special algebraic identities}
$(a+b)^2=a^2+2ab+b^2$\\
$(a-b)^2=a^2-2ab+b^2$\\
$(a+b)(a-b)=a^2-b^2$



\section{Equations}

\subsection{Quadratic formula}

$x=\dfrac{-b\pm\sqrt{b^2-4ac}}{2a}$



\section{Set Language \& Notation}

$\in$: is an element of\\
n($A$): number of elements in set $A$\\
$A'$: complement of set A\\
$\varnothing$: empty set\\
$\xi$: universal set\\
$\cup$: union\\
$\cap$: intercept\\
$\subset$: subset	


\section{Problems In Real-World Contexts}

\subsection{Simple interest}

$I=\dfrac{PRT}{100}$

\subsection{Compound interest}

$A=P\left(1+\dfrac{R}{100}\right)^n$


\section{Angles, Triangles \& Polygons}



\subsection{Types of polygons}
\begin{table}[H]
\centering
\begin{tabularx}{0.2\textwidth}{|Y|Y|}
\hline
\heading{No. of sides}& \heading{Polygons} \\\hline
3 & triangle \\\hline
4 & quadrilateral \\\hline
5 & \textbf{pent}agon \\\hline
6 & \textbf{hex}agon \\\hline
7 & \textbf{hept}agon \\\hline
8 & \textbf{oct}agon \\\hline
9 & \textbf{non}agon \\\hline
10 & \textbf{dec}agon \\\hline
\end{tabularx}
\end{table}

\subsection{Sum of interior \& exterior angles}

Sum of interior angles $=(n-2)\times180\degree$\\
Sum of exterior angles $=360\degree$



\section{Congruence \& Similarity}


\subsection{Congruent \& Similar Triangles}

\begin{center}
\begin{tabularx}{0.28\textwidth}{|Y|Y|}
\hline
\textbf{Congruent triangles} & \textbf{Similar triangles}\\\hline
SSS, SAS, AAS, RHS & SSS, SAS, AAA\\\hline
\end{tabularx}
\end{center}

\subsection{Ratio of area \& volume}

$\dfrac{A_1}{A_2}=\left(\dfrac{l_1}{l_2}\right)^2$\\
$\dfrac{V_1}{V_2}=\left(\dfrac{l_1}{l_2}\right)^3$


\section{Pythagoras' Theorem \& Trigonometry}

\subsection{Pythagoras' theorem}

$a^2+b^2=c^2$

\subsection{Trigonometric ratios}

$\tan\theta=\frac{\text{opposite}}{\text{adjacent}}$\\
$\cos\theta=\frac{\text{adjacent}}{\text{hypotenuse}}$\\
$\sin\theta=\frac{\text{opposite}}{\text{hypotenuse}}$\\

\begin{center}
\textbf{TOA CAH SOH }is applicable for only right-angled triangles
\end{center}


\subsection{Obtuse angles}
$\sin(180\degree-\theta)=\sin\theta$\\
$\cos(180\degree-\theta)=-\cos\theta$

\subsection{Sine rule}
$\dfrac{a}{\sin A}=\dfrac{b}{\sin B}$


\subsection{Cosine rule}
$c^2=a^2+b^2-2 ab \cos C$

\subsection{Area of triangle}

$\text{Area of triangle}=\frac{1}{2}ab \sin C$

\subsection{Bearings}

\begin{figure}[H]
\centering
\psset{xunit=0.7cm,yunit=0.7cm,algebraic=true,dimen=middle,dotstyle=o,dotsize=3pt 0,linewidth=0.8pt,arrowsize=3pt 2,arrowinset=0.25}
\begin{pspicture}(8,-3.33291)(16.42291,3.33291)
\psline[linewidth=0.8pt,arrowsize=0.05291667cm 2.0,arrowlength=1.4,arrowinset=0.4]{<-}(12.021015,2.9145117)(12.021015,-2.8454883)
\psline[linewidth=0.8pt,arrowsize=0.05291667cm 2.0,arrowlength=1.4,arrowinset=0.4]{<-}(14.901015,0.014511718)(9.141016,0.014511718)
\begin{scriptsize}
\usefont{T1}{ptm}{m}{n}
\rput(12.112471,3.1195116){$000\degree$}
\usefont{T1}{ptm}{m}{n}
\rput(12.01247,-3.1604884){$180\degree$}
\usefont{T1}{ptm}{m}{n}
\rput(15.332471,0.048828){$090\degree$}
\usefont{T1}{ptm}{m}{n}
\rput(8.71247,0.019511718){$270\degree$}
\end{scriptsize}
\end{pspicture} 
\end{figure}

A bearing is a \textbf{3-digit positive number} with units of degree to show direction \textbf{clockwise} from the north direction.


\section{Mensuration}

\subsection{Conversion}
$1\text{ m}= 100\text{ cm}$\\
$1\text{ m}^2= 10,000\text{ cm}^2$\\
$1\text{ m}^3= 1,000,000\text{ cm}^3$\\


\subsection{Radian \& Degree}
$180\degree = \pi \text{ rad}$\\

\subsection{Arc length \& sector area}

\subsubsection{Degree}

$s = \frac{\theta}{360\degree} \times 2 \pi r$, where $\theta$ is in degrees\\
$A = \frac{\theta}{360\degree} \times \pi r^2$, where $\theta$ is in degrees

\subsubsection{Radian}

$s = r \theta$, where $\theta$ is in radians\\
$A = \frac{1}{2} r^2 \theta$, where $\theta$ is in radians

\columnbreak

\section{Coordinate Geometry}

\subsection{Cartesian coordinate}

$(x,y)$
\subsection{Gradient}
{$m=\dfrac{y_1-y_2}{x_1-x_2}$}


\subsection{Equation}
$y-y_1=m(x-x_1)$\\
$y=mx+c$\\
*Vertical line: $x=a$\\
*Horizontal line: $y=b$

\subsection{Length}
$\text{Length } = \sqrt{(x_2 - x_1)^2 + (y_2 - y_1)^2}$




\section{Vectors In 2 Dimensions}

\subsection{Representation}
Vectors can be represented by $\begin{pmatrix}x\\y\end{pmatrix}$, $\vv{AB}$, \textbf{a} or $\utilde{a}$.

\subsection{Magnitude}
$|\vv{AB}|\text{ or }|\textbf{a}| = \sqrt{x^2+y^2}$


\section{Data Analysis}

\subsection{Mode}
Mode is the \textbf{most frequently occurring} number. A set of data can have \textbf{more than one} mode.

\subsection{Mean}
$\text{mean} = \dfrac{\text{sum of all numbers}}{\text{number of numbers}}$

\subsection{Median}
Median is the \textbf{centre} number when the numbers are arranged from \textbf{smallest to largest}.

\subsection{Range}
Range = maximum $-$ minimum


\subsection{Quartiles \& percentiles}

\begin{figure}[H]
\centering
\psscalebox{0.55 0.55}{
\begin{pspicture}(1,-1.0238281)(16.936583,1.0238281)
\psline[linewidth=0.04cm](2.2269921,-0.0061328127)(14.3469925,-0.0061328127)
\psline[linewidth=0.04cm](8.286992,0.4338672)(8.286992,-0.4261328)
\psline[linewidth=0.04cm](5.266992,0.4338672)(5.266992,-0.4261328)
\psline[linewidth=0.04cm](11.306993,0.41386718)(11.306993,-0.4461328)
 \rput(8.314707,0.79886717){median}
 \rput(5.2264256,0.8388672){lower quartile}
 \rput(11.331612,0.79886717){upper quartile}
 \rput(4.96792,-0.7811328){25\textsuperscript{th} percentile}
 \rput(8.303174,-0.7811328){50\textsuperscript{th} percentile}
 \rput(11.305107,-0.76113284){75\textsuperscript{th} percentile}
\psdots[dotsize=0.12](2.2069921,-0.026132813)
\psdots[dotsize=0.12](14.366992,-0.026132813)
 \rput(2.255283,-0.8011328){0\textsuperscript{th} percentile}
 \rput(14.420693,-0.7411328){100\textsuperscript{th} percentile}
 \rput(14.354717,0.8188672){maximum}
 \rput(2.244717,0.8388672){minimum}
\end{pspicture} 
}
\end{figure}


\subsection{Interquartile range}
Interquartile range = upper quartile $-$ lower quartile

\subsection{Box-and-whisker plot}

\begin{figure}[H]
\centering
\psscalebox{0.45 0.45}{
\begin{pspicture}(0,-1.878496)(17.26,1.8384961)
\psline[linewidth=0.04cm](11.42,1.1584961)(14.42,1.1584961)
\psline[linewidth=0.04cm](2.26,-0.8015039)(2.26,-1.3215039)
 \rput(7.467715,-1.6265039){median}
 \rput(5.2394335,-1.6565039){lower quartile}
 \rput(11.244619,-1.6565039){upper quartile}
\psdots[dotsize=0.12](2.28,1.138496)
\psdots[dotsize=0.12](14.44,1.138496)
 \rput(2.2777245,-1.6565039){minimum}
\psline[linewidth=0.04cm,arrowsize=0.05291667cm 2.0,arrowlength=1.4,arrowinset=0.4]{->}(0.0,-1.0415039)(17.24,-1.0415039)
\psline[linewidth=0.04cm](14.44,-0.7815039)(14.44,-1.3015039)
 \rput(14.447724,-1.6565039){maximum}
\psframe[linewidth=0.04,dimen=outer](11.42,1.8384961)(5.82,0.4584961)
\psline[linewidth=0.04cm](2.24,1.138496)(5.86,1.1184961)
\psline[linewidth=0.04cm](5.82,-0.8015039)(5.82,-1.3215039)
\psline[linewidth=0.04cm](7.44,-0.8215039)(7.44,-1.3415039)
\psline[linewidth=0.04cm](11.42,-0.8015039)(11.42,-1.3215039)
\psline[linewidth=0.04cm](7.44,1.8184961)(7.44,0.4584961)
\end{pspicture} 
}
\end{figure}

\subsection{Mean \& standard deviation}

\subsubsection{Ungrouped}
$\text{Mean}, \bar{x} =\frac{\sum x}{N}$\\
$\text{Standard deviation}, \sigma =\sqrt{\frac{\sum(x-\overline{x})^2}{N}}$




\subsubsection{Grouped}

$\text{Mean}, \overline{x}=\frac{\sum fx}{\sum f}$\\
$\text{Standard deviation}, \sigma =\sqrt{\frac{\sum fx^2}{\sum f}-\left(\frac{\sum fx}{\sum f}\right)^2}$





\end{multicols}




\end{document}